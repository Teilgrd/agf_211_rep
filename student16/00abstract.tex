
\begin{abstract}
	As part of the annual AGF-211 fieldwork, St. \ Jonsforden was visited in April 2018, to perform in situ measurements of newly formed sea ice, and snow ice above. Wire harps was installed in forming sea ice, and in snow on top of the already established sea ice. The data showed a clear and coherent evolution of the salinity down in the sea, and of the brine through the snow. The newly formed sea ice showed most of its brine release when the ice was growing, with a maximum of $50 \text{g}/\text{kg} \pm 1.76\text{g}/\text{kg}$ at the beginning. The snow experiment was successful in showing the rise of brine, both trough flooding and through capillary rise. A new model of capillary rise was developed, and tuned to the data. 
%Using an improved version of the wire harp the hope this year was to gather one coherent set of data, which would provide new insight into how salinity affects the snow above, and it develops in newly formed sea ice. With one harp installed partly in the sea, and the other harp on the ice, covered in snow at a location with negative free-board, this years findings showed a clear and coherent evolution of the salinity both down in the sea, and up through the snow. The salinity evolution in the newly formed sea ice showed the most release of brine during the periods where we had the most ice growth with the maximum concentration of $50 \text{g}/\text{kg} \pm 1.76\text{g}/\text{kg}$ at the beginning. The Snow experiment showed clearly that the rise of brine happens in two different steps; one step through flooding, the other through capillary rise. Both happening at very different rates, with flooding developing fast and the capillary rise developing slow. 
    

	%used for the past two years to study sea ice growth, and its salinity evolution. 
	%Where, when, what (but not why)
	%state principal objectives and scope of investigation
	%quick presentation of the methods employed
	%summarize the results and findings, give specific numbers of our findings
\end{abstract}
% single paragraph, past tense, no references

% state principal objectives and scope of investigation

% describe methods employed

% summarize the results

% state the principal conclusions

%------
%'Exeeds Standard' criteria for Report Grading:
%Abstract is concise, highly informative without unnecessary detail, complete, and easy to understand.
%------
