
\clearpage

\section{Introduction}

%Where, when,\\
%map of svalbard with overview over st. jones fjorden and location of the experiment in relation to boat

%Why:\\
An important aspect to climate models is the understanding of sea ice, two important yet not well understood aspects to this is the growth of new ice and the formation of snow ice. In climate models snow is often just used as an insulator, and as a factor in the albedo. If snow ice formation is included, it is only in the form of flooding. Snow ice is formed from refreezing of wet snow, this can happen due to some melt event, or brine from the sea ice coming up. The easiest way to explain this phenomena is when flooding occurs. That is when the amount of snow is large enough, so that pushes the ice into the sea, thus creating a negative free-board. Brine can also enter the snow via capillary rise. Earlier studies in capillary rise have mostly focused on the hight it can reach and not the velocity it rises with, they have also focused on using density as their main factor and not tried to understand pore size \textcite{Capill}.\\
\\
Furthermore, if the amount of annual precipitation continues to grow, as suggested in \textcite{Forland}, this will become more and more important in the coming years for the Svalbard sea ice. Fichfet and Maquada has studied the sensitivity of global sea ice models, and states that it is important to implement snow ice formation \textcite{Fichefet}. \\
\\
In addition to this, newly formed sea ice and especially the brine evolution of sea ice is still poorly understood, and has previously mostly been based around ice core drilling. A method which is destructive, show a poor time resolution, and does not exactly provide reliable data as there is always brine escaping the core when its removed from the ice. \textcite{Fuchs}. \textcite{Notz} used a self made wire harp to study thin ice growth, this tool has been further developed through several steps, the most recent one used in \textcite{Fuchs}. Here the harp was used to measure brine evolution in both growing sea ice and snow.\\
\\
For researching we joined the AGF-211 cruise to St. \ Jonsfjorden, located centrally on the west-coast of Spitsbergen in the Svalbard Archipelago. The fieldwork took place in the time period of 15/04-18 \-- 22/04-18.

%Set the present work into a scientific, regional, and then global perspective\\
%In the introduction you must refer to some central published papers\\
%Motivation: Why are we doing this? Why should the reader continue to read this paper?\\
%Dirks thesis, and the master thesis done by Matt.\\
%Yannic also had a something with ideas for motiation\\

%Refrences!!\\

%Present the structure of the report\\

%During a boat trip to St. Jonsfjorden with AGF-211, during 15.04.18 - 21.04.18 we used a devolping instrument called the wire harp, to set up experiments in growing thin ice and in snow ice.


%present tense, include references
%------
%'Exeeds Standard' criteria for Report Grading:
%The Introduction is engaging, clearly states the main topic, covers any necessary background material and theory, and previews the structure of the paper.
%------
